\documentclass[11pt]{article}
\usepackage[margin=1in]{geometry}
\usepackage{enumerate}
\usepackage{framed}
\usepackage{graphicx}
\usepackage{bm}
\usepackage{amssymb}
\usepackage{amsmath}
\usepackage{multicol}
\usepackage{hyperref}
\begin{document}

\begin{verbatim}
"""
Author      : Savannah Baron and Varsha Kishore
Class       : HMC CS 158
Date        : 2017 Feb 27
Description : Utilities for Famous Faces
"""
"""
Author      : Yi-Chieh Wu
Class       : HMC CS 158
Date        : 2017 Feb 27
Description : Famous Faces
"""

# python libraries
import collections

# numpy libraries
import numpy as np

# matplotlib libraries
import matplotlib.pyplot as plt

# libraries specific to project
import util
from cluster import *

######################################################################
# helper functions
######################################################################

def build_face_image_points(X, y) :
    """
    Translate images to (labeled) points.
    
    Parameters
    --------------------
        X     -- numpy array of shape (n,d), features (each row is one image)
        y     -- numpy array of shape (n,), targets
    
    Returns
    --------------------
        point -- list of Points, dataset (one point for each image)
    """
    
    n,d = X.shape
    
    images = collections.defaultdict(list) # key = class, val = list of images with this class
    for i in xrange(n) :
        images[y[i]].append(X[i,:])
    
    points = []
    for face in images :
        count = 0
        for im in images[face] :
            points.append(Point(str(face) + '_' + str(count), face, im))
            count += 1

    return points


def plot_clusters(clusters, title, average) :
    """
    Plot clusters along with average points of each cluster.

    Parameters
    --------------------
        clusters -- ClusterSet, clusters to plot
        title    -- string, plot title
        average  -- method of ClusterSet
                    determines how to calculate average of points in cluster
                    allowable: ClusterSet.centroids, ClusterSet.medoids
    """
    
    plt.figure()
    np.random.seed(20)
    label = 0
    colors = {}
    centroids = average(clusters)
    for c in centroids :
        coord = c.attrs
        plt.plot(coord[0],coord[1], 'ok', markersize=12)
    for cluster in clusters.members :
        label += 1
        colors[label] = np.random.rand(3,)
        for point in cluster.points :
            coord = point.attrs
            plt.plot(coord[0], coord[1], 'o', color=colors[label])
    plt.title(title)
    plt.show()


def generate_points_2d(N, seed=1234) :
    """
    Generate toy dataset of 3 clusters each with N points.
    
    Parameters
    --------------------
        N      -- int, number of points to generate per cluster
        seed   -- random seed
    
    Returns
    --------------------
        points -- list of Points, dataset
    """
    np.random.seed(seed)
    
    mu = [[0,0.5], [1,1], [2,0.5]]
    sigma = [[0.1,0.1], [0.25,0.25], [0.15,0.15]]
    
    label = 0
    points = []
    for m,s in zip(mu, sigma) :
        label += 1
        for i in xrange(N) :
            x = util.random_sample_2d(m, s)
            points.append(Point(str(label)+'_'+str(i), label, x))
    
    return points


######################################################################
# k-means and k-medoids
######################################################################

def random_init(points, k) :
    """
    Randomly select k unique elements from points to be initial cluster centers.
    
    Parameters
    --------------------
        points         -- list of Points, dataset
        k              -- int, number of clusters
    
    Returns
    --------------------
        initial_points -- list of k Points, initial cluster centers
    """
    ### ========== TODO : START ========== ###
    # part 2c: implement (hint: use np.random.choice)
    return np.random.choice(points, size=k, replace=False)
    ### ========== TODO : END ========== ###


def cheat_init(points) :
    """
    Initialize clusters by cheating!
    
    Details
    - Let k be number of unique labels in dataset.
    - Group points into k clusters based on label (i.e. class) information.
    - Return medoid of each cluster as initial centers.
    
    Parameters
    --------------------
        points         -- list of Points, dataset
    
    Returns
    --------------------
        initial_points -- list of k Points, initial cluster centers
    """
    ### ========== TODO : START ========== ###
    # part 2e: implement
    initial_points = []
    pointsDict = {}
    clusterSet = ClusterSet()

    # Find cluster of each point based on labels
    for point in points:
        if point.label in pointsDict:
            pointsDict[point.label].append(point)
        else:
            pointsDict[point.label] = [point]
    # Add each found cluster to a cluster set
    for pointsSet in pointsDict.values():
        clusterSet.add(Cluster(pointsSet))
    return clusterSet.medoids()
    ### ========== TODO : END ========== ###

def definitelyNotCheating_init(points, k):
    """
    Initialize clusters by definitely not cheating!
    (AKA we find the distance from one point to all others,
    and then pick points at even distances from that point
    to try to get a good distribution in space)
    
    Parameters
    --------------------
        points         -- list of Points, dataset
        k              -- int, number of clusters
    
    Returns
    --------------------
        initial_points -- list of k Points, initial cluster centers
    """
    initial_points = []
    # Pick a point randomly to start on
    startPoint = np.random.choice(points)
    # Calculate distance from point to all other points
    distances = [np.linalg.norm(startPoint.attrs-other.attrs) for other in points]
    sortedDistances = np.sort(distances)
    # Calculate k evenly spaced intervals in sorted array
    interval = int(np.floor(len(points) / float(k)))
    for i in range(0, len(points), interval):
        # For each interval append a point at that distance to the initial points
        pointIndex = np.where(sortedDistances[i] == distances)[0][0]
        initial_points.append(points[pointIndex])
    return initial_points

def findNearestAveLabel(point, aves):
    """
    Return which average index a point is nearest to
    Parameters
    --------------------
        point  --  Point
        centroid       -- list of Points, centroids of clusters
    
    Returns
    --------------------
        i -- index of nearest cluster
    """
    return np.argmin([point.distance(ave) for ave in aves])

def kAverages(points, k, average, init='random', plot=True):
    """
    Cluster points into k clusters using variations of k-means algorithm.
    
    Parameters
    --------------------
        points  -- list of Points, dataset
        k       -- int, number of clusters
        average -- method of ClusterSet
                   determines how to calculate average of points in cluster
                   allowable: ClusterSet.centroids, ClusterSet.medoids
        init    -- string, method of initialization
                   allowable: 
                       'cheat'  -- use cheat_init to initialize clusters
                       'random' -- use random_init to initialize clusters
        plot    -- bool, True to plot clusters with corresponding averages
                         for each iteration of algorithm
    
    Returns
    --------------------
        k_clusters -- ClusterSet, k clusters
    """
    
    ### ========== TODO : START ========== ###
    # part 2c: implement
    # Hints:
    #   (1) On each iteration, keep track of the new cluster assignments
    #       in a separate data structure. Then use these assignments to 
    #       create new Cluster objects and a new ClusterSet object. Then
    #       update the centroids.
    #   (2) Repeat until the clustering no longer changes.
    #   (3) To plot, use plot_clusters(...).
    # Initialize cluster centers:
    aves = []
    if init == "random":
        aves = random_init(points, k)
    elif init == "cheat":
        aves = cheat_init(points)
    elif init == "notCheat":
        aves = definitelyNotCheating_init(points, k)
    prev_aves = []
    k_clusters = ClusterSet()
    it = 0
    #equivalent
    while not np.array_equal([ave.attrs for ave in aves], prev_aves):
        k_clusters = ClusterSet()
        clusterPoints = [[] for ave in aves]
        # Assign points to nearest centroid
        for point in points:
            i = findNearestAveLabel(point, aves)
            clusterPoints[i].append(point)
        # Move cluster centroid to mean of points assigned
        for pointList in clusterPoints:
            cluster = Cluster(pointList)
            k_clusters.add(cluster)
        prev_aves = [ave.attrs for ave in aves]
        aves = average(k_clusters)
        if plot:
            plot_clusters(k_clusters, "Iteration ".format(iter)+ str(it), average)
        it += 1
    return k_clusters
    ### ========== TODO : END ========== ###

def kMeans(points, k, init='random', plot=False) :
    """
    Cluster points into k clusters using variations of k-means algorithm.
    
    Parameters
    --------------------
        points  -- list of Points, dataset
        k       -- int, number of clusters
        average -- method of ClusterSet
                   determines how to calculate average of points in cluster
                   allowable: ClusterSet.centroids, ClusterSet.medoids
        init    -- string, method of initialization
                   allowable: 
                       'cheat'  -- use cheat_init to initialize clusters
                       'random' -- use random_init to initialize clusters
        plot    -- bool, True to plot clusters with corresponding averages
                         for each iteration of algorithm
    
    Returns
    --------------------
        k_clusters -- ClusterSet, k clusters
    """
    
    ### ========== TODO : START ========== ###
    # part 2c: implement
    # Hints:
    #   (1) On each iteration, keep track of the new cluster assignments
    #       in a separate data structure. Then use these assignments to 
    #       create new Cluster objects and a new ClusterSet object. Then
    #       update the centroids.
    #   (2) Repeat until the clustering no longer changes.
    #   (3) To plot, use plot_clusters(...).
    # Initialize cluster centers:
    return kAverages(points, k, ClusterSet.centroids, init, plot)
    ### ========== TODO : END ========== ###


def kMedoids(points, k, init='random', plot=False) :
    """
    Cluster points in k clusters using k-medoids clustering.
    See kMeans(...).
    """
    ### ========== TODO : START ========== ###
    # part 2d: implement
    return kAverages(points, k, ClusterSet.medoids, init, plot)
    ### ========== TODO : END ========== ###


######################################################################
# main
######################################################################

def main() :
    ### ========== TODO : START ========== ###
    # part 1: explore LFW data set
    X, y = util.get_lfw_data()
    n, d = X.shape
    util.show_image(X[500, :])
    util.show_image(X[1000, :])
    util.show_image(X[1500, :])

    mean = np.mean(X, axis=0)
    util.show_image(mean)

    U, mu = util.PCA(X)
    util.plot_gallery([util.vec_to_image(U[:,i]) for i in xrange(12)])

    l_list = [1, 10, 50, 100, 500, 1288]
    for l in l_list:
        Z, Ul = util.apply_PCA_from_Eig(X, U, l, mu)
        X_rec = util.reconstruct_from_PCA(Z, Ul, mu)
        util.plot_gallery([util.vec_to_image(X_rec[i,:]) for i in xrange(12)])
    ### ========== TODO : END ========== ###
    
    
    
    #========================================
    # part 2
    
    # part b: test Cluster implementation    
    # centroid: [ 1.04022358  0.62914619]
    # medoid:   [ 1.05674064  0.71183522]
    
    np.random.seed(1234)
    sim_points = generate_points_2d(20)
    cluster = Cluster(sim_points)
    print 'centroid:', cluster.centroid().attrs
    print 'medoid:', cluster.medoid().attrs
    
    # parts 2c-e: test kMeans and kMedoids implementation using toy dataset
    np.random.seed(1234)
    sim_points = generate_points_2d(20)
    k = 3
    
    # cluster using random initialization
    kmeans_clusters = kMeans(sim_points, k, init='random', plot=True)
    kmedoids_clusters = kMedoids(sim_points, k, init='random', plot=True)
    
    # cluster using cheat initialization
    kmeans_clusters = kMeans(sim_points, k, init='cheat', plot=True)
    kmedoids_clusters = kMedoids(sim_points, k, init='cheat', plot=True)    
    
    
    
    ### ========== TODO : START ========== ###    
    # part 3a: cluster faces
    np.random.seed(1234)
    X1, y1 = util.limit_pics(X, y, [4, 6, 13, 16], 40)
    points = build_face_image_points(X1, y1)
    kmeans_score_list = []
    kmedoids_score_list = []
    k = 4
    for i in xrange(10):
        points = build_face_image_points(X1, y1)
        kmeans_clusters = kMeans(points, k, init='random', plot=False)
        kmedoids_clusters = kMedoids(points, k, init='random', plot=False) 
        kmeans_score_list.append(kmeans_clusters.score())
        kmedoids_score_list.append(kmedoids_clusters.score())
    print "Average for k-means: ", np.mean(kmeans_score_list)
    print "Minimum for k-means: ", min(kmeans_score_list)
    print "Maximum for k-means: ", max(kmeans_score_list)
    print "Average for k-medoids: ", np.mean(kmedoids_score_list)
    print "Minimum for k-medoids: ", min(kmedoids_score_list)
    print "Maximum for k-medoids: ", max(kmedoids_score_list)

    # part 3b: explore effect of lower-dimensional representations on clustering performance
    np.random.seed(1234)
    k = 2
    X1, y1 = util.limit_pics(X, y, [4, 13], 40)
    U, mu = util.PCA(X1)
    kmeans_score_list = []
    kmedoids_score_list = []
    for l in xrange(1, 42, 2):
        Z, Ul = util.apply_PCA_from_Eig(X1, U, l, mu)
        points = build_face_image_points(Z, y1)
        kmeans_clusters = kMeans(points, k, init='cheat', plot=False)
        kmedoids_clusters = kMedoids(points, k, init='cheat', plot=False) 
        kmeans_score_list.append(kmeans_clusters.score())
        kmedoids_score_list.append(kmedoids_clusters.score())
    plt.plot(range(1,42,2), kmedoids_score_list, 'go', label="k-Medoids")
    plt.plot(range(1,42,2), kmeans_score_list, 'bo', label="k-Means")
    plt.legend()
    plt.show()

    # part 3c: determine ``most discriminative'' and ``least discriminative'' pairs of images
    np.random.seed(1234)

    # Find "average" face for each class of faces
    labels = np.unique(y)
    averages = []
    for label in labels:
        # Find images with that label
        indices = np.where(y==label)
        pointVals = X[indices]
        # Average across all images in class
        averagePoint = np.mean(pointVals, axis=0)
        averages.append(averagePoint)

    # Find further and closest faces based on averages
    farDist = 0
    farFaces = None
    closeDist = float("inf")
    closeFaces = None
    # Compare all distances
    for i in range(len(labels)):
        for j in range(i+1, len(labels)):
            dist = np.linalg.norm(averages[i]-averages[j])
            if dist > farDist:
                farDist = dist
                farFaces = (i, j)
            if dist < closeDist:
                closeDist = dist
                closeFaces = (i, j)
    # Pull out images for each chosen face
    closeLabel1 = closeFaces[0]
    closeIndices1 = np.where(y==closeLabel1)[0]
    closeLabel2 = closeFaces[1]
    closeIndices2 = np.where(y==closeLabel2)[0]
    farLabel1 = farFaces[0]
    farIndices1 = np.where(y==farLabel1)[0]
    farLabel2 = farFaces[1]
    farIndices2 = np.where(y==farLabel2)[0]

    print "Close: ", closeLabel1
    util.plot_gallery([util.vec_to_image(X[i,:]) for i in closeIndices1])
    print "Close: ", closeLabel2
    util.plot_gallery([util.vec_to_image(X[i,:]) for i in closeIndices2])
    print "Far: ", farLabel1
    util.plot_gallery([util.vec_to_image(X[i,:]) for i in farIndices1])
    print "Far: ", farLabel2
    util.plot_gallery([util.vec_to_image(X[i,:]) for i in farIndices2])

    close_score_list = []
    far_score_list = []
    k = 2
    # Get points for each chosen face and run k-Medoids
    # k-Medoids chosen because it performed better than k-Means in previous runs
    XClose, yClose = util.limit_pics(X, y, [closeLabel1, closeLabel2], 40)
    XFar, yFar = util.limit_pics(X, y, [farLabel1, farLabel2], 40)
    pointsClose = build_face_image_points(XClose, yClose)
    pointsFar = build_face_image_points(XFar, yFar)
    for i in range(10):
        kmedoids_clusters = kMedoids(pointsClose, k, init='random', plot=False) 
        close_score_list.append(kmedoids_clusters.score())
        kmedoids_clusters = kMedoids(pointsFar, k, init='random', plot=False) 
        far_score_list.append(kmedoids_clusters.score())
    print "Average for close: ", np.mean(close_score_list)
    print "Minimum for close: ", min(close_score_list)
    print "Maximum for close: ", max(close_score_list)
    print "Average for far: ", np.mean(far_score_list)
    print "Minimum for far: ", min(far_score_list)
    print "Maximum for far: ", max(far_score_list)

    # part 4: Test out our new init!
    np.random.seed(1234)
    X1, y1 = util.limit_pics(X, y, [4, 6, 13, 16], 40)
    points = build_face_image_points(X1, y1)
    kmeans_score_list = []
    kmedoids_score_list = []
    k = 4
    for i in xrange(10):
        points = build_face_image_points(X1, y1)
        kmeans_clusters = kMeans(points, k, init='notCheat', plot=False)
        kmedoids_clusters = kMedoids(points, k, init='notCheat', plot=False) 
        kmeans_score_list.append(kmeans_clusters.score())
        kmedoids_score_list.append(kmedoids_clusters.score())
    print "Average for k-means: ", np.mean(kmeans_score_list)
    print "Minimum for k-means: ", min(kmeans_score_list)
    print "Maximum for k-means: ", max(kmeans_score_list)
    print "Average for k-medoids: ", np.mean(kmedoids_score_list)
    print "Minimum for k-medoids: ", min(kmedoids_score_list)
    print "Maximum for k-medoids: ", max(kmedoids_score_list)
    
    ### ========== TODO : END ========== ###


if __name__ == "__main__" :
    main()
 \end{verbatim}   
 \end{document}